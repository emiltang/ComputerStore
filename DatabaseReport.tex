\documentclass[a4paper]{article}
\usepackage[utf8]{inputenc}
\usepackage[danish]{babel}
\usepackage[T1]{fontenc}
\usepackage{noto}
\usepackage{sourcecodepro}
\usepackage{fullpage}
\usepackage[dvipsnames]{xcolor}
\usepackage{listings}
\usepackage{setspace}
\onehalfspacing
\usepackage{graphicx}

\lstset{
    language=sql,
    basicstyle=\footnotesize\ttfamily,
    breaklines=true,
    showtabs=false,
    showspaces=false,
    breakatwhitespace=true,
    numbers=left,
    captionpos=b,
    numberstyle=\footnotesize\ttfamily,
    keywordstyle=\bfseries\color{PineGreen},
    identifierstyle=\color{RoyalBlue},
    commentstyle=\color{green},
    stringstyle=\color{red}
}

\title{\textbf{Database Design og Programming Eksamens Projekt}}
\author{Emil Tang Kristensen}
\date{23. marts 2017}

\begin{document}
\maketitle
\tableofcontents

\section{Introduktion}
Denne rapport dokumenterer det individuelle eksamensprojekt i faget \emph{Database Design og Programmering}, som en del af 2. semester på uddannelsen \emph{Software Engineering} der udbydes af \emph{Syddansk Universitet}.
Projekts formål er at udvikle et produkt i form af en java applikation med en tilhørende postgresql database.
Produktet skal være et salg og lager system til et forretning der sælger computer hardware.
Mere information om kravene til systemet kan findes i projektbeskrivelsen.

\section{Entity Relationship Model}
Modellen kan ses på figur \ref{fig:er} i bilag \ref{appendix:er}

\section{Implemtation}

\subsection{Salg af Computer Systemer}
Før at et computer system kan sælges, checkes først om der er nok komponenter til at bygge som minimum et system. 
Sql-koden til dette check vist på listing \ref{lst:checkComputerSystem},
Informationerne bliver hentet fra et left join mellem \texttt{ComputerSystems} og \texttt{Components}. Relationerne joines på \texttt{ComputerSystems} arritibuter der indeholder et component id og er lig med et id i \texttt{Components}.


\begin{lstlisting}[caption={Check om der er nok komponenter til at sælge et system},label={lst:checkComputerSystem}]
SELECT min(currentStock)
FROM ComputerSystems LEFT JOIN Components ON
ComputerSystems.cpu = Components.id OR
ComputerSystems.ram = Components.id OR
ComputerSystems.mainBoard = Components.id OR
ComputerSystems.computerCase = Components.id OR
ComputerSystems.gpu = Components.id
WHERE ComputerSystems.id = 123
GROUP BY ComputerSystems.id
\end{lstlisting}

\begin{lstlisting}[caption=Salg af et computer system]
UPDATE Components
SET currentStock = currentStock - 1
FROM ComputerSystems
WHERE ComputerSystems.id = 123
\end{lstlisting}

\section{Brugermanual}
Programmet startes og accepterer input via terminalen eller output konsollen.
\begin{itemize}
    \item Input karekteren \texttt{0} for at vise en liste over alle komponenter og tilhørende lagerantal.
    \item Input \texttt{1} for at vise alle computer systemer og hvor mange der kan bygges af hver.
    \item Input \texttt{2} for at vise en pris liste over alle komponenter og computer systemer.
    \item Input \texttt{3} og derfter et computer system identifikationsnummer samt et antal af systemer, for at få vist et pris tilbud.
    \item Input \texttt{4} for at sælge et computer system eller en komponent.

        \begin{itemize}
            \item Input \texttt{0} og et id for at sælge en komponent.
            \item Input \texttt{1} og et id for at sælge et computer system.
        \end{itemize}

    \item Input \texttt{5} for at vise en liste over komponenter der skal indkøbes flere af og hvor mange der skal købes.
\end{itemize}

\section{Tests}

\begin{thebibliography}{9}

    \bibitem{Ullman}
    Jeffrey D. Ullman, Hector Garcia-Molina, Jennifer Widom,
    \emph{Database Systems the Complete Book Second Edition},
    Department of Computer Science Stanford University,
    Pearson.

    \bibitem{Liang}
    Y. Daniel Liang,
    \emph{Introduction to Java Programming Comprehensive Version 10. Edition},
    Armstrong Atlantic State University,
    Pearson.

\end{thebibliography}

\appendix

\section{Entity Relationship Model}
\label{appendix:er}

\begin{figure}[htbp!]
    \centering
    \includegraphics[width=\textwidth]{ER}
    \caption{Entity Realtionship Model}
    \label{fig:er}
\end{figure}

\section{Realtioner}

\begin{lstlisting}[caption=\texttt{Components} Relation]
CREATE TABLE Components (
    id              INTEGER PRIMARY KEY,
    name            TEXT,
    price           REAL,
    minStock        INTEGER,
    prefStock       INTEGER,
    currentStock    INTEGER
);
\end{lstlisting}

\begin{lstlisting}[caption=Sub-relationer til \texttt{Components}]
CREATE TABLE CPUs (
    id              INTEGER PRIMARY KEY REFERENCES Components (id),
    busSpeed        INTEGER,
    socket          CPUsockets
);
CREATE TABLE RAMs (
    id              INTEGER PRIMARY KEY REFERENCES Components (id),
    busSpeed        INTEGER,
    ramType         RAMTypes
);
CREATE TABLE GPUs (
    id              INTEGER PRIMARY KEY REFERENCES Components (id)
);
CREATE TABLE Mainboards (
    id              INTEGER PRIMARY KEY REFERENCES Components (id),
    hasOnBoardGPU   BOOLEAN,
    socket          CPUsockets,
    formFactor      FormFactors,
    ramType         RAMTypes,
    busSpeed        INTEGER
);
CREATE TABLE ComputerCases (
    id              INTEGER PRIMARY KEY REFERENCES Components (id),
    formFactor      FormFactors
);
\end{lstlisting}

\begin{lstlisting}[caption=\texttt{ComputerSystem} Relation]
CREATE TABLE ComputerSystems (
    id              INTEGER PRIMARY KEY,
    name            TEXT,
    cpu             INTEGER REFERENCES CPUs (id),
    mainboard       INTEGER REFERENCES Mainboards (id),
    gpu             INTEGER REFERENCES GPUs (id),
    ram             INTEGER REFERENCES RAMs (id),
    computerCase    INTEGER REFERENCES ComputerCases (id)
);
\end{lstlisting}

\end{document}

